%%% LaTeX Template: Designer's CV
%%%
%%% Source: http://www.howtotex.com/
%%% Feel free to distribute this template, but please keep the referal to HowToTeX.com.
%%% Date: March 2012

%%%%%%%%%%%%%%%%%%%%%%%%%%%%%%%%%%%%%
% Document properties and packages
%%%%%%%%%%%%%%%%%%%%%%%%%%%%%%%%%%%%%
\documentclass[a4paper,12pt,final]{memoir}

% misc
\renewcommand{\familydefault}{bch}	% font
\pagestyle{empty}					% no pagenumbering
\setlength{\parindent}{0pt}			% no paragraph indentation


% required packages (add your own)
\usepackage{flowfram}										% column layout
\usepackage[top=1cm,left=1cm,right=1cm,bottom=1cm]{geometry}% margins
\usepackage{graphicx}										% figures
\usepackage{url}	
\usepackage[utf8]{inputenc}
% URLs
\usepackage[usenames,dvipsnames]{xcolor}					% color
\usepackage{paralist}										% compact lists
\usepackage{multicol}                                       % columns env.
    \setlength{\multicolsep}{0pt}
\usepackage{tikz}

%%%%%%%%%%%%%%%%%%%%%%%%%%%%%%%%%%%%%
% Create column layout
%%%%%%%%%%%%%%%%%%%%%%%%%%%%%%%%%%%%%
% define length commands
\setlength{\vcolumnsep}{\baselineskip}
\setlength{\columnsep}{\vcolumnsep}

% frame setup (flowfram package)
% left frame
\newflowframe{0.2\textwidth}{\textheight}{0pt}{0pt}[left]
	\newlength{\LeftMainSep}
	\setlength{\LeftMainSep}{0.2\textwidth}
	\addtolength{\LeftMainSep}{1\columnsep}
 
% small static frame for the vertical line
\newstaticframe{1.5pt}{\textheight}{\LeftMainSep}{0pt}
 
% content of the static frame
\begin{staticcontents}{1}
\hfill
\tikz{%
	\draw[loosely dotted,color=RoyalBlue,line width=1.5pt,yshift=0]
	(0,0) -- (0,\textheight);}%
\hfill\mbox{}
\end{staticcontents}
 
% right frame
\addtolength{\LeftMainSep}{1.5pt}
\addtolength{\LeftMainSep}{1\columnsep}
\newflowframe{0.7\textwidth}{\textheight}{\LeftMainSep}{0pt}[main01]


%%%%%%%%%%%%%%%%%%%%%%%%%%%%%%%%%%%%%
% define macros (for convience)
%%%%%%%%%%%%%%%%%%%%%%%%%%%%%%%%%%%%%
\newcommand{\Sep}{\vspace{1.5em}}
\newcommand{\SmallSep}{\vspace{0.5em}}

\newenvironment{Ommeg}
	{\ignorespaces\textbf{\color{RoyalBlue} Om meg}}
	{\Sep\ignorespacesafterend}
	
\newcommand{\CVSection}[1]
	{\Large\textbf{#1}\par
	\SmallSep\normalsize\normalfont}

\newcommand{\CVItem}[1]
	{\textbf{\color{RoyalBlue} #1}}


%%%%%%%%%%%%%%%%%%%%%%%%%%%%%%%%%%%%%
% Begin document
%%%%%%%%%%%%%%%%%%%%%%%%%%%%%%%%%%%%%
\begin{document}

% Left frame
%%%%%%%%%%%%%%%%%%%%
\begin{figure}
	\hfill
	\includegraphics[width=0.6\columnwidth]{photo.jpg}
	\vspace{-7cm}
\end{figure}

\begin{flushright}\small
	Håkon Ø. Løvdal \\
	\url{me@hakloev.no}  \\
	\url{https://hakloev.no} \\
	+47 940 50 008
\end{flushright}\normalsize
\framebreak


% Right frame
%%%%%%%%%%%%%%%%%%%%
\Huge\bfseries {\color{RoyalBlue} Håkon Ødegård Løvdal} \\
\Large\bfseries  Informatikkstudent \\

\normalsize\normalfont

% About me
\begin{Ommeg}
Er positiv og endringsvillig. Anser meg selv som utadvendt. Lærer raskt og jobber godt både selvstendig og med andre. Tar ansvar for egen læring og er god til å anvende ny kunnskap. Har erfaring med ledelse av grupper og store mengder med materiell. 
\end{Ommeg}

% Education
\CVSection{Utdanning}
\CVItem{2012 - nå, Norges teknisk-naturvitenskaplige universitet}\\
Bachelor i informatikk
\SmallSep

\CVItem{2008 - 2011, Nøtterøy Videregående Skole}\\
Språk, samfunnsfag og økonomi
\Sep

% Experience
\CVSection{Arbeidserfaring}
\CVItem{Sommer 2013 og 2014, Spar Tjøme}\\
\textit{Varelassleder}, ansvar for butikkstandard, kundebehandling, varepåfylling og annet forefallende butikkarbeid.
\SmallSep

\CVItem{2011 - 2012, Forsvaret, Hæren}\\
\textit{Visekorporal, nestlagfører}, ansvarlig for utstyrskontroll over troppsmatriell og ledelse av matriellag. Fungerte også som stormingeniør.
\SmallSep

\CVItem{2009 - 2011, Rimi Tjøme}\\
\textit{Skiftleder}, ansvar for butikkstandard, kundebehandling, varepåfylling og annet forefallende butikkarbeid. Fungerte også som butikksjefs stedfortreder.
\Sep

% Verv
\CVSection{Verv}
\CVItem{2013 - nå, Online, linjeforeningen for informatikk}\\
\textit{Komitémedlem - Ekskursjonskomiteen}, Ekskursjonskomiteen har ansvaret for å arrangere informatikkstudentene i tredje årstrinn sin årlige ekskursjon. Ekskursjonen er av akademisk grad og går gjerne til et sted av utenlandsk opprinnelse hvor næringsliv og utdanningsinstitusjoner blir besøkt.
\SmallSep

\CVItem{2013 - nå, Online, linjeforeningen for informatikk}\\
\textit{Komitémedlem - Bank- og økonomikomiteen}, Bank- og økonomikomiteen er ansvarlig for alt det økonomiske innad i Online. Dette består av regnskapsøring, fakturering og å ha kontroll på det som skjer i nettbanken. Representant for Fag- og kurskomiteen.
\SmallSep

\CVItem{2012 - nå, Online, linjeforeningen for informatikk}\\
\textit{Komitémedlem - Fag- og kurskomiteen}, Fag- og kurskomiteen jobber med å arrangere kurs, workshop, faglige presentasjoner og andre faglige arrangementer for studentene ved informatikk.
\Sep

\clearpage
\framebreak
\framebreak

% Skills
\CVSection{Kompetanse}

\CVItem{Nøkkelkvalifikasjoner}
\begin{multicols}{3}
\begin{compactitem}[\color{RoyalBlue}$\circ$]
    \item Endringsvillig 
    \item Utadvendt
    \item Selvstendig
    \item Tar ansvar
    \item Positiv
    \item Lærer raskt
\end{compactitem}
\end{multicols}
\SmallSep

\CVItem{Plattformer}
\begin{multicols}{3}
\begin{compactitem}[\color{RoyalBlue}$\circ$]
    \item OS X 
    \item Linux
    \item Unix/FreeBSD
\end{compactitem}
\end{multicols}
\SmallSep

\CVItem{Datakompetanse}
\begin{multicols}{3}
\begin{compactitem}[\color{RoyalBlue}$\circ$]
	\item Python 
	\item Java
	\item Latex 
	\item HTML
	\item CSS
	\item nginx 
	\item MySQL 
	\item Microsoft Office
\end{compactitem}
\end{multicols}
\SmallSep 

\CVItem{Interesser}
\begin{multicols}{3}
\begin{compactitem}[\color{RoyalBlue}$\circ$]
	\item Politikk 
	\item Historie
	\item Ølbrygging 
	\item Musikk
	\item Trening
	\item Venner
	\item Linjeforeningen
\end{compactitem}
\end{multicols}
\Sep 

% References
\CVSection{Referenaser}
Referanser oppgis ved behov

%%%%%%%%%%%%%%%%%%%%%%%%%%%%%%%%%%%%%
% End document
%%%%%%%%%%%%%%%%%%%%%%%%%%%%%%%%%%%%%
\end{document}
